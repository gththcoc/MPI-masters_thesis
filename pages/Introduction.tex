\addcontentsline{toc}{part}{Introduction}

\chapter*{Introduction}

Gravitational waves (GWs) are extremely weak space-time perturbations predicted by Albert Einstein in 1916\cite{Einstein:1916cc}. Their effects were first observed in Hulse and Taylor's binary system PSR B1913+16 in 1974, whose orbital decay data coincided with a binary system emitting gravitational radiation\cite{Weisberg:1981mt}. A few decades later, in 2015, the LIGO interferometers achieved the first direct measurement of gravitational waves produced by a BBH system\cite{LIGOScientific:2016aoc}. As of the writing of this thesis, dozens of events produced mainly by CBCs have been detected without EM counterparts. The only exception so far has been the event GW170817\cite{LIGOScientific:2017vwq}, detected by LIGO and Virgo. An extensive campaign of EM observations was launched and confirmed its coincidence with a BNS merger at around 40MPc in the galaxy NGC 4993. It produced EM transients such as GRB170817A and AT 2017gfo caused by heavy elements in the mass ejection, their radioactive decay, and interaction with the interstellar medium\cite{Abbott_2017, LIGOScientific:2017ync, 10.1093/mnras/stz1564}. Unfortunately, unlike the EM transients produced during the postmerger stage, such as GRBs and the kilonova afterglow, the postmerger phase of gravitational waves was lost in the instrumental noise\cite{LIGOScientific:2018hze, LIGOScientific:2018urg, LIGOScientific:2017fdd}. 

To reach the highest sensitivity existing GW search pipelines use a modeled search algorithm called matched filtering. The algorithm uses prior knowledge about the astrophysical signal and detector sensitivity estimates to extract GWs of known morphology from frequency-dependent noisy data streams of known power spectral density(PSD)\cite{Usman:2015kfa, Sachdev:2019vvd, Aubin:2020goo}. However, it is well known that the computational costs when running the matched filtering algorithm increase when our theoretical models for a specific GW have a high dimensional parameter space \cite{Allen_2021, Dhurkunde:2021csz}.

This thesis investigates, in particular, BNS postmerger GWs in the context of data analysis. Due to the complex morphology of such waves \cite{Maggiore:2018sht, Shibata:2019wef,Radice_2020}, most existing state-of-the-art phenomenological waveform models have not been successful in reproducing such a signal with few free parameters\cite{Breschi:2019srl, Tsang:2019esi, Soultanis:2021oia, https://doi.org/10.48550/arxiv.2205.09112}. For this reason, rather than using such state-of-the-art phenomenological waveform models, the matched filtering algorithm is run iteratively over a set of 220 publicly available BNS numerical relativity waveforms to estimate how much SNR can be recovered from them at a distance of 100MPc using known detector PSDs and simple models like a finite monochromatic model with constant amplitude and finite monochromatic model with zero crossing amplitude envelope.

Chapter \ref{gravphys} reviews key concepts in Newtonian gravity and general relativity, establishing the importance of some physical observables relevant to GW astronomy. Chapter \ref{BNS-merg} gives an overview of what is known about BNS mergers, their postmerger phases, and currently available waveform catalogs. Chapter \ref{DA} describes the matched filtering algorithm, its frequency domain representation, and the technical details of its implementation using fast Fourier transforms (FFTs). Finally, Chapter \ref{res} introduces the different template models used in the matched filtering run and shows the results over the whole waveform dataset.


