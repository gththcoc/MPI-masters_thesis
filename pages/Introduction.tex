%\pagestyle{mystyle}

\addcontentsline{toc}{part}{Introduction}

\chapter*{Introduction}
Gravitational waves are extremely weak space-time perturbations predicted by A. Einstein in 1916\cite{Einstein:1916cc}. Their effects were first observed in Hulse and Taylor's binary system PSR B1913+16 in 1974, whose orbital decay data coincided with a binary system emitting gravitational radiation\cite{Weisberg:1981mt}. A few decades later, in 2015, LIGO interferometers achieved the first direct measurement of gravitational waves produced by a BBH system\cite{LIGOScientific:2016aoc}. As of the writing of this thesis, dozens of events produced mainly by CBCs have been detected without EM counterparts. The only exception so far has been the event GW170817\cite{LIGOScientific:2017vwq}, for which a large campaign of EM observations was launched that confirmed its coincidence with a BNS merger at around 40MPc in the galaxy NGC 4993. It produced EM transients such as GRB170817A and AT 2017gfo caused by heavy elements in the mass ejection, their radioactive decay, and interaction with the interstellar medium\cite{Abbott_2017, LIGOScientific:2017ync, 10.1093/mnras/stz1564}. Unfortunately, unlike the EM transients produced during the postmerger stage, such as GRBs and the afterglow, the postmerger phase of gravitational waves was lost in the instrumental noise\cite{LIGOScientific:2018hze, LIGOScientific:2018urg, LIGOScientific:2017fdd}. 

In recent years, detectors such as Virgo and KAGRA with similar sensitivities have joined the collaboration, enhancing the capabilities of the detector network for future observing runs. Their interferometers provide data dominated by frequency-dependent noise, often featuring non-gaussian transients that can mask signals of astrophysical origin, as happened with GW170817 in LIGO Livingston\cite{LIGOScientific:2017vwq}. Such technical difficulties have been addressed during the development of search pipelines; some use methods that look for excess power in the time-frequency plane\cite{Drago:2020kic}, and others use matched filtering to achieve higher sensitivity \cite{Usman:2015kfa, Sachdev:2019vvd, Aubin:2020goo}.  

The matched filtering algorithm combines our best gravitational wave models and estimates of detector noise through their PSDs to obtain the optimal linear filter \cite{Creighton:2011zz, Wainstein:1962vrq}. However, it is well known that its computational costs increase with the length of the signals and number of free parameters of the model \cite{Allen_2021, Dhurkunde:2021csz}, which becomes a problem when looking for very long signals \cite{Sathyaprakash:2009xs} or very complex morphologies.

This thesis investigates in particular BNS postmerger GWs in the context of data analysis, using the matched filtering algorithm. Since they have not been detected, numerical relativity simulations are currently our best tool to study the postmerger evolution and extract the gravitational radiation form them. Due to their complex morphology\cite{Maggiore:2018sht, Shibata:2019wef,Radice_2020}, most existing state-of-the-art pheno waveform models have not been successful in reproducing such a signal with few free parameters\cite{Breschi:2019srl, Tsang:2019esi, Soultanis:2021oia, https://doi.org/10.48550/arxiv.2205.09112}. For this reason about 200 Numerical relativity waveforms have been taken from publicly available catalogs to estimate how much SNR can be recovered from them using the matched filtering algorithm and simple models at a distance of 100MPc and several detector PSDs.

Chapter\ref{gravphys} reviews key concepts of gravitational physics in Newtonian gravity and general relativity and it stablish a parallelism between both theories around physical obsevables relevant in GW astronomy. Chapter\ref{BNS-merg} gives an overview of what is known about BNS mergers, their postmerger phases and currently available waveform catalogs. Chapter\ref{DA} describes the matched filtering algorithm, its frequency domain representation, and the technical details of its implementation using fast Fourier tranforms(FFTs). Finally, Chapter\ref{res} shows the results obtained over all waveforms in the catalogs using several template models.
