\newpage
\chapter*{Conventions}
In this work, relativistic computations use SI units unless is otherwise stated. The values of the speed of light in vacuum\cite{si-brochure}, gravitational constant\cite{Anderson_2015} and solar mass are taken up to four significant figures.

$$ c = 2.997\times 10^8 \frac{m}{s}, $$

$$ G = 6.674\times 10^{-11}\frac{m^3}{kg\cdot s^2},$$

$$ M_{\odot} = 1.988 \times 10^{30} kg.$$

The space-time metric $g$ is taken with the signature "mostly minus" so that, in an orthonormal frame, with basis vectors $e_{\alpha}$, and dual basis vectors $\theta^{\alpha}$, it has the form:

$$ g = g_{\alpha \beta} \;\; \theta^{\alpha} \otimes \theta^{\beta} = diag(+1,-1,-1,-1)\;\; ,$$

and timelike vectors have a positive norm. Einstein's summation convention is assumed if an index appears repeated as one upper index, and one lower index or viceversa.

$$ Z^\mu Y_\mu = \sum_{\mu=0}^3 Z^\mu Y_\mu $$

Widely known terminology in the context of gravitational wave astronomy will be abbreviated using the following acronyms:

\vspace{1cm}

GW: Gravitational wave

EOS: Equation-of-state

CBC: Compact binary coalescence

NS: Neutron star

BNS: Binary neutron star

BBH: Binary black hole

BHNS: Black hole-neutron star 

SNR: Signal-to-noise ratio

EM: Electromagnetic

PSD: Power spectral density




