\chapter*{Abstract}

With the rise of gravitational wave astronomy and multimessenger astrophysics, the postmerger phase of binary neutron star (BNS) coalescences has gained much attention. Numerical relativity (NR) simulations have shown that gravitational waves (GWs) and electromagnetic signatures generated during this late phase are complex and highly dependent on the equation of state of supranuclear matter. This thesis addresses one of the main problems related to state-of-the-art BNS GW postmerger waveform models: a high-dimensional parameter space makes them computationally expensive when used in modeled search pipelines. Given simple postmerger GW models and a set of 220 NR BNS waveforms, we use the matched filtering algorithm to compute the signal to noise ratio (SNR) and examine whether such models can be used to estimate the postmerger's dominant frequency and duration. We find that a finite monochromatic model with constant amplitude recovers 60-80\% of the optimal SNR for postmerger GW signals generated by equal-mass non-spinning binaries while closely recovering the waveform's duration and dominant frequency. Further, a finite monochromatic model with a zero-crossing amplitude envelope is used to model the postmerger first amplitude minimum. Using such a model, we observe marginal differences in the SNR recovery in some waveforms,  and the postmerger's duration and dominant frequency estimation improves.








\vspace{1cm}
\textbf{Keywords:} Gravitational-waves; Postmerger; Binary-Neutron-Star;Numerical-relativity; Matched-Filtering.
