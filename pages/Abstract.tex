\chapter*{Abstract}

With the rise of gravitational wave astronomy and multimessenger astrophysics, the postmerger phase of binary neutron star(BNS) coalescences has recently gained much attention. Numerical relativity(NR) simulations have shown that gravitational waves(GWs) and electromagnetic signatures during this late phase are complex and highly dependent on the equation of state of supranuclear matter, which makes their observation not only interesting to the astrophysics community but the nuclear physics community as well. This thesis addresses one of the main problems related to existing modeled GW search pipelines when it comes to BNS gravitational waves: existing postmerger phenomenological models have a high-dimensional parameter space. Hence low latency search pipelines can not use BNS GW approximants that include the postmerger phase since it becomes computationally expensive. Simple postmerger models are used to see how well they perform in terms of signal-to-noise-ratio (SNR) recovery when overlapped against a set of 220 publicly available numerical relativity BNS waveforms using the matched filtering algorithm. We find a finite monochromatic model with constant amplitude recovers 60-80\% of the optimal SNR for postmerger GW signals generated by equal-mass non-spinning binaries while closely recovering the waveform's duration and main frequency $f_2$. Similar amounts of SNR can be recovered for other types of waveforms, but the waveform's duration and main frequency are not accurately recovered. Further, a finite monochromatic model with zero-crossing amplitude envelope is used to model the postmerger first amplitude minimum. We observe that modeling such feature improves the postmerger's duration and main frequency recovery, but only marginal improvements to the SNR recovery are found for some waveforms.







\vspace{1cm}
\textbf{Keywords:} Gravitational-waves; Postmerger; Binary-Neutron-Star;Numerical-relativity; Matched-Filtering.
